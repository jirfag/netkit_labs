\documentclass[a4paper,12pt]{article}

\input{header.tex}

\title{Отчёт по лабораторной работе \\ <<Динамическая IP-маршрутизация>>}
\author{Исаев Д.С}

\begin{document}

\maketitle

\tableofcontents

\section{Настройка сети}

\subsection{Топология сети}

Топология сети и используемые IP-адреса показаны на рисунке~\ref{fig:network}.

\begin{figure}
\centering
\includegraphics[width=0.8\textwidth]{includes/network_gv.pdf}
\caption{Топология сети}
\label{fig:network}
\end{figure}

Перечень узлов, на которых используется динамическая IP-маршрутизация: ...


\subsection{Назначение IP-адресов}

Ниже приведён файл сетевой настройки  маршрутизатора r1.

\begin{Verbatim}
auto lo
iface lo inet loopback

auto eth0
iface eth0 inet static
    address 10.0.10.1
    netmask 255.255.255.0

auto eth1
iface eth1 inet static
    address 10.0.30.1
    netmask 255.255.255.0

auto eth2
iface eth2 inet static
    address 10.0.50.1
    netmask 255.255.255.0
\end{Verbatim}

Ниже приведён файл сетевой настройки рабочей станции wsp1.

\begin{Verbatim}
auto lo
iface lo inet loopback

auto eth0
iface eth0 inet static
    address 10.0.10.3
    netmask 255.255.255.0

gateway 10.0.10.1
\end{Verbatim}


\subsection{Настройка протокола RIP}

Ниже приведен файл \Code{/etc/quagga/ripd.conf} маршрутизатора r1.

\begin{Verbatim}
! Этот настройки, касающиеся протокола RIP.
router rip

! Раскомментируйте ниже все интерфейсы, подключённые
! к сетям с другими маршрутизаторами.
network eth0
network eth1
network eth2

! Уменьшаем значения всех таймеров для ускорения опытов.
! Рассылка: 10 сек., устаревание: 60 cек., сборка мусора: 120 сек.
timers basic 10 60 120

! Следующие две строчки заставляют маршрутизатор
! добавлять в сообщения протокола RIP все известные ему маршруты.
redistribute kernel
redistribute connected

! Это имя файла журнала службы RIP.
! Его содержимое можно изучить в случае неполадок
log file /var/log/quagga/ripd.log
\end{Verbatim}


Ниже приведен файл \Code{/etc/quagga/ripd.conf} рабочий станции, связанной с несколькими маршрутизаторами wsp1.

\begin{Verbatim}
! Этот настройки, касающиеся протокола RIP.
router rip

! Раскомментируйте ниже все интерфейсы, подключённые
! к сетям с другими маршрутизаторами.
network eth0
! network eth1
! network eth2

! Уменьшаем значения всех таймеров для ускорения опытов.
! Рассылка: 10 сек., устаревание: 60 cек., сборка мусора: 120 сек.
timers basic 10 60 120

! Следующие две строчки заставляют маршрутизатор
! добавлять в сообщения протокола RIP все известные ему маршруты.
redistribute kernel
redistribute connected

! Это имя файла журнала службы RIP.
! Его содержимое можно изучить в случае неполадок
log file /var/log/quagga/ripd.log
\end{Verbatim}


\section{Проверка настройки протокола RIP}

Вывод \textbf{traceroute} от узла r1 до wsp2 при нормальной работе сети.

\begin{Verbatim}
r1:~# traceroute 10.0.60.3
traceroute to 10.0.60.3 (10.0.60.3), 64 hops max, 40 byte packets
 1  10.0.30.2 (10.0.30.2)  6 ms  0 ms  0 ms
 2  10.0.60.3 (10.0.60.3)  11 ms  1 ms  0 ms
\end{Verbatim}

Вывод \textbf{traceroute} от узла такого-то до внешнего IP 195.19.38.2.

\begin{Verbatim}
r4:~# traceroute 195.19.38.2
traceroute to 195.19.38.2 (195.19.38.2), 64 hops max, 40 byte packets
 1  172.16.0.1 (172.16.0.1)  1 ms  1 ms  1 ms
 2  192.168.0.1 (192.168.0.1)  2 ms  1 ms  5 ms
 3  10.183.192.1 (10.183.192.1)  3 ms  2 ms  2 ms
 4  213.85.208.250 (213.85.208.250)  7 ms  3 ms  3 ms
 5  193.232.244.44 (193.232.244.44)  65 ms  5 ms  92 ms
 6  194.85.40.117 (194.85.40.117)  4 ms  5 ms  4 ms
 7  194.190.254.106 (194.190.254.106)  5 ms  5 ms  6 ms
 8  * * *
\end{Verbatim}

Вывод сообщения RIP.

\begin{Verbatim}
r1:~# tcpdump -i eth2 -tvn -s 1518 udp
tcpdump: listening on eth2, link-type EN10MB (Ethernet), capture size 1518 bytes
IP (tos 0x0, ttl 1, id 0, offset 0, flags [DF], proto UDP (17), length 132) 10.0.50.1.520 > 224.0.0.9.520:
        RIPv2, Response, length: 104, routes: 5
          AFI: IPv4:       10.0.10.0/24, tag 0x0000, metric: 1, next-hop: self
          AFI: IPv4:       10.0.20.0/24, tag 0x0000, metric: 2, next-hop: self
          AFI: IPv4:       10.0.30.0/24, tag 0x0000, metric: 1, next-hop: self
          AFI: IPv4:       10.0.40.0/24, tag 0x0000, metric: 2, next-hop: self
          AFI: IPv4:       10.0.60.0/24, tag 0x0000, metric: 2, next-hop: self
\end{Verbatim}

Вывод таблицы RIP.

\begin{Verbatim}
r1# show ip rip
Codes: R - RIP, C - connected, S - Static, O - OSPF, B - BGP
Sub-codes:
      (n) - normal, (s) - static, (d) - default, (r) - redistribute,
      (i) - interface

     Network            Next Hop         Metric From            Tag Time
R(n) 0.0.0.0/0          10.0.50.2             2 10.0.50.2         0 00:50
C(i) 10.0.10.0/24       0.0.0.0               1 self              0
R(n) 10.0.20.0/24       10.0.10.2             2 10.0.10.2         0 00:58
C(i) 10.0.30.0/24       0.0.0.0               1 self              0
R(n) 10.0.40.0/24       10.0.10.2             2 10.0.10.2         0 00:58
C(i) 10.0.50.0/24       0.0.0.0               1 self              0
R(n) 10.0.60.0/24       10.0.30.2             2 10.0.30.2         0 00:56
\end{Verbatim}

Вывод таблицы маршрутизации.

\begin{Verbatim}
r1:~# ip r
10.0.20.0/24 via 10.0.10.2 dev eth0  proto zebra  metric 2
10.0.50.0/24 dev eth2  proto kernel  scope link  src 10.0.50.1
10.0.60.0/24 via 10.0.30.2 dev eth1  proto zebra  metric 2
10.0.30.0/24 dev eth1  proto kernel  scope link  src 10.0.30.1
10.0.40.0/24 via 10.0.10.2 dev eth0  proto zebra  metric 2
10.0.10.0/24 dev eth0  proto kernel  scope link  src 10.0.10.1
default via 10.0.50.2 dev eth2  proto zebra  metric 2
\end{Verbatim}

\section{Расщепленный горизонт и испорченные обратные обновления}

Включенное расщепление горизонта.
\begin{Verbatim}
r1:~# tcpdump -i eth2 -tvn -s 1518 udp
tcpdump: listening on eth2, link-type EN10MB (Ethernet), capture size 1518 bytes
IP (tos 0x0, ttl 1, id 0, offset 0, flags [DF], proto UDP (17), length 132) 10.0.50.1.520 > 224.0.0.9.520:
        RIPv2, Response, length: 104, routes: 5
          AFI: IPv4:       10.0.10.0/24, tag 0x0000, metric: 1, next-hop: self
          AFI: IPv4:       10.0.20.0/24, tag 0x0000, metric: 2, next-hop: self
          AFI: IPv4:       10.0.30.0/24, tag 0x0000, metric: 1, next-hop: self
          AFI: IPv4:       10.0.40.0/24, tag 0x0000, metric: 2, next-hop: self
          AFI: IPv4:       10.0.60.0/24, tag 0x0000, metric: 2, next-hop: self
\end{Verbatim}

Включенное расщепление горизонта с включенными испорченными обновлениями.
\begin{Verbatim}
r1:~# tcpdump -i eth2 -tvn -s 1518 udp
tcpdump: listening on eth2, link-type EN10MB (Ethernet), capture size 1518 bytes
IP (tos 0x0, ttl 1, id 0, offset 0, flags [DF], proto UDP (17), length 172) 10.0.50.1.520 > 224.0.0.9.520:
        RIPv2, Response, length: 144, routes: 7
          AFI: IPv4:         0.0.0.0/0 , tag 0x0000, metric: 16, next-hop: 10.0.50.2
          AFI: IPv4:       10.0.10.0/24, tag 0x0000, metric: 1, next-hop: self
          AFI: IPv4:       10.0.20.0/24, tag 0x0000, metric: 2, next-hop: self
          AFI: IPv4:       10.0.30.0/24, tag 0x0000, metric: 1, next-hop: self
          AFI: IPv4:       10.0.40.0/24, tag 0x0000, metric: 2, next-hop: self
          AFI: IPv4:       10.0.50.0/24, tag 0x0000, metric: 16, next-hop: self
          AFI: IPv4:       10.0.60.0/24, tag 0x0000, metric: 2, next-hop: self
\end{Verbatim}

Отключенное расщепление горизонта.
\begin{Verbatim}
IP (tos 0x0, ttl 1, id 0, offset 0, flags [DF], proto UDP (17), length 172) 10.0.50.1.520 > 224.0.0.9.520:
        RIPv2, Response, length: 144, routes: 7
          AFI: IPv4:         0.0.0.0/0 , tag 0x0000, metric: 2, next-hop: 10.0.50.2
          AFI: IPv4:       10.0.10.0/24, tag 0x0000, metric: 1, next-hop: self
          AFI: IPv4:       10.0.20.0/24, tag 0x0000, metric: 2, next-hop: 10.0.50.2
          AFI: IPv4:       10.0.30.0/24, tag 0x0000, metric: 1, next-hop: self
          AFI: IPv4:       10.0.40.0/24, tag 0x0000, metric: 2, next-hop: self
          AFI: IPv4:       10.0.50.0/24, tag 0x0000, metric: 1, next-hop: self
          AFI: IPv4:       10.0.60.0/24, tag 0x0000, metric: 2, next-hop: self
\end{Verbatim}


\section{Имитация устранимой поломки в сети}

Выключили маршрутизатор r3.
Вывод таблицы RIP непосредственно перед истечением таймера устаревания (на маршрутизаторе-соседе отключенного r4).

\begin{Verbatim}
r4# show ip rip
Codes: R - RIP, C - connected, S - Static, O - OSPF, B - BGP
Sub-codes:
      (n) - normal, (s) - static, (d) - default, (r) - redistribute,
      (i) - interface

     Network            Next Hop         Metric From            Tag Time
K(r) 0.0.0.0/0          172.16.0.1            1 self              0
R(n) 10.0.10.0/24       10.0.50.1             2 10.0.50.1         0 00:58
C(i) 10.0.20.0/24       0.0.0.0               1 self              0
R(n) 10.0.30.0/24       10.0.50.1             2 10.0.50.1         0 00:58
R(n) 10.0.40.0/24       10.0.60.1             2 10.0.60.1         0 00:02
C(i) 10.0.50.0/24       0.0.0.0               1 self              0
C(i) 10.0.60.0/24       0.0.0.0               1 self              0
\end{Verbatim}

Перестроенная таблица на этом же маршрутизаторе

\begin{Verbatim}
r4# show ip rip
Codes: R - RIP, C - connected, S - Static, O - OSPF, B - BGP
Sub-codes:
      (n) - normal, (s) - static, (d) - default, (r) - redistribute,
      (i) - interface

     Network            Next Hop         Metric From            Tag Time
K(r) 0.0.0.0/0          172.16.0.1            1 self              0
R(n) 10.0.10.0/24       10.0.50.1             2 10.0.50.1         0 00:54
C(i) 10.0.20.0/24       0.0.0.0               1 self              0
R(n) 10.0.30.0/24       10.0.50.1             2 10.0.50.1         0 00:54
R(n) 10.0.40.0/24       10.0.60.1            16 10.0.60.1         0 01:58
C(i) 10.0.50.0/24       0.0.0.0               1 self              0
C(i) 10.0.60.0/24       0.0.0.0               1 self              0
\end{Verbatim}


Вывод \textbf{traceroute} от узла r4 до r2 (10.0.40.1) после того, как служба RIP перестроила таблицы маршрутизации.

\begin{Verbatim}
r4:~# traceroute 10.0.40.1
traceroute to 10.0.40.1 (10.0.40.1), 64 hops max, 40 byte packets
 1  10.0.40.1 (10.0.40.1)  0 ms  2 ms  0 ms
\end{Verbatim}

\section{Имитация неустранимой поломки в сети}

Выключили маршрутизатор r2.
\begin{Verbatim}
r4# show ip rip
Codes: R - RIP, C - connected, S - Static, O - OSPF, B - BGP
Sub-codes:
      (n) - normal, (s) - static, (d) - default, (r) - redistribute,
      (i) - interface

     Network            Next Hop         Metric From            Tag Time
K(r) 0.0.0.0/0          172.16.0.1            1 self              0
R(n) 10.0.10.0/24       10.0.50.1            16 10.0.50.1         0 01:01
C(i) 10.0.20.0/24       0.0.0.0               1 self              0
R(n) 10.0.30.0/24       10.0.50.1            16 10.0.50.1         0 01:01
R(n) 10.0.40.0/24       10.0.20.1            16 10.0.20.1         0 01:38
C(i) 10.0.50.0/24       0.0.0.0               1 self              0
C(i) 10.0.60.0/24       0.0.0.0               1 self              0
\end{Verbatim}

\begin{Verbatim}
IP (tos 0x0, ttl 1, id 0, offset 0, flags [DF], proto UDP (17), length 52) 10.0.50.2.520 > 224.0.0.9.520:
        RIPv2, Response, length: 24, routes: 1
          AFI: IPv4:       10.0.10.0/24, tag 0x0000, metric: 16, next-hop: self
IP (tos 0x0, ttl 1, id 0, offset 0, flags [DF], proto UDP (17), length 52) 10.0.60.2.520 > 224.0.0.9.520:
        RIPv2, Response, length: 24, routes: 1
          AFI: IPv4:       10.0.10.0/24, tag 0x0000, metric: 16, next-hop: self
IP (tos 0x0, ttl 1, id 0, offset 0, flags [DF], proto UDP (17), length 132) 10.0.50.2.520 > 224.0.0.9.520:
        RIPv2, Response, length: 104, routes: 5
          AFI: IPv4:         0.0.0.0/0 , tag 0x0000, metric: 1, next-hop: self
          AFI: IPv4:       10.0.10.0/24, tag 0x0000, metric: 16, next-hop: self
          AFI: IPv4:       10.0.20.0/24, tag 0x0000, metric: 1, next-hop: self
          AFI: IPv4:       10.0.40.0/24, tag 0x0000, metric: 16, next-hop: self
          AFI: IPv4:       10.0.60.0/24, tag 0x0000, metric: 1, next-hop: self
IP (tos 0x0, ttl 1, id 0, offset 0, flags [DF], proto UDP (17), length 132) 10.0.60.2.520 > 224.0.0.9.520:
        RIPv2, Response, length: 104, routes: 5
          AFI: IPv4:         0.0.0.0/0 , tag 0x0000, metric: 1, next-hop: self
          AFI: IPv4:       10.0.10.0/24, tag 0x0000, metric: 16, next-hop: self
          AFI: IPv4:       10.0.20.0/24, tag 0x0000, metric: 1, next-hop: self
          AFI: IPv4:       10.0.40.0/24, tag 0x0000, metric: 16, next-hop: self
          AFI: IPv4:       10.0.50.0/24, tag 0x0000, metric: 1, next-hop: self
\end{Verbatim}


\end{document}
